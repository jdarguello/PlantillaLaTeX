\begin{center}
	\section{Preprocesamiento}
\end{center}

\noindent
\justify

La etapa de \textit{preprocesamiento} de la planta de extracci\'on consiste en la disminuci\'on del tama\~no de part\'icula del material a procesar. Para tal efecto, se seleccion\'o el \textbf{molino de bolas} en la etapa de dise\~no conceptual. 

\begin{figure}[h!]
	\centering
	\includegraphics[width=\textwidth]{Images/Pre/CAD.jpeg}
	\caption{Sistema de preprocesamiento.}
	\label{prepro}
\end{figure}

\noindent
\justify

El sistema de molienda, mostrado en la Figura \ref{prepro}, consiste del molino, un sistema de transmisi\'on y una estructura met\'alica. El \textit{molino} contendr\'a el material vegetal seco, el agente dispersante y las bolas. El \textit{sistema de transmisi\'on} consiste de un motor el\'ectrico, ejes y chumaceras. La \textit{estructura met\'alica} cumple la funci\'on de soportar las cargas, tanto din\'amicas como est\'aticas, transmitidas durante la etapa de molienda. 

\newpage

\noindent
\justify

El espacio que ocupar\'a este sistema se puede apreciar en la Figura \ref{molino}.

\begin{figure}[h!]
	\centering
	\includegraphics[width=\textwidth]{Images/Pre/plano.jpeg}
	\caption{Espacio ocupado por el sistema de preprocesamiento.}
	\label{molino}
\end{figure} 
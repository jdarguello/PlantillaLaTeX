\begin{center}
	\section{Sistema de molienda}
\end{center}

\noindent
\justify

Durante la etapa de dise\~no funcional del molino de bolas, se definieron las siguientes dimensiones y par\'ametros operacionales:

\begin{itemize}
	\item Volumen de $0.633 \left[m ^3 \right]$.
	\item Di\'ametro de $0.813 [m]$.
	\item Longitud de $1.22 [m]$.
	\item Velocidad de rotaci\'on de $37.5 [rpm]$.
	\item Potencia m\'inima de operaci\'on $3.36 [hp]$.
\end{itemize}

\vspace{-1mm}

\noindent
\justify

Se realiz\'o una evaluaci\'on num\'erica para determinar la mejor configuraci\'on de las aletas del molino y tama\~nos de bola (ver Figura \ref{aletas}) que garantizan la mayor efectividad posible, a trav\'es de la evaluaci\'on de la \textit{energ\'ia de impacto} de las bolas y su \textit{distribuci\'on} sobre el molino. Para ello, se desarrollar\'o una simulaci\'on num\'erica, empleando el m\'etodo\footnote{Los m\'etodos num\'ericos son teoremas matem\'aticos que permiten describir la naturaleza de diferentes fen\'omenos de car\'acter f\'isico-qu\'imico. Son ampliamente usados en ingenier\'ia como metodolog\'ias predictivas durante el proceso de dise\~no funcional y mec\'anico.} de elementos discretos (DEM), validada mediante una metodolog\'ia te\'orica simplificada.

\begin{figure}[h!]
\centering
\begin{tikzpicture}
	%Molino
	\draw (0,0) circle (3.1cm);
	\draw[dashed, blue!50] (0,-4) -- (0,-2.5);
	\draw[dashed, blue!50] (0,0) -- (0,4);
	\draw[dashed, blue!50] (-4,0) -- (0,0);
	\draw[dashed, blue!50] (2.5,0) -- (4,0);
	\draw (-0.51,2.945) arc(-260:-190:3);
	\draw[red] (-0.51,2.945) -- (0,2.5) -- (0.51,2.945);
	\draw[red] (-2.945,0.51) -- (-2.5,0) -- (-2.945, -0.51);
	\draw (-2.945, -0.51) arc(-170:-100:3);
	\draw[red] (-0.51,-2.945) -- (0,-2.5) -- (0.51,-2.945);
	\draw (0.51, -2.945) arc(-80:-10:3);
	\draw[red] (2.945,0.51) -- (2.5,0) -- (2.945, -0.51);
	\draw (2.945,0.51) arc(10:80:3);
	
	\draw[blue!50!red, arrows={-Triangle[angle=90:5pt,blue!50!red,fill=blue!50!red]}] (3.4,0.8) arc(20:60:4.5) node [midway, above=2.6mm, right=0.2mm] {$\overrightarrow{T}$};
	
	%División volumétrica
	\draw[dashed, step=0.5cm, blue!60!green, pattern=north west lines, pattern color=blue!60!green] (-2.05,-2.05) -- (2.05,2.05) arc(45:12:3) -- (2.4, 0) -- (2.85,-0.55) arc(-12:-78:3) -- (0,-2.4) -- (-0.55,-2.85) arc(-102:-135:3) -- cycle;
	
	%Bolas
	\draw[dashed] plot [smooth] coordinates { (2.05,2.05) (1,2) (-1,1) (-2,-0.8) (-2.05,-2.05) };
	\draw[fill=blue!70!black] (-1,1) circle (1.2mm);
	\draw[blue!70!black, arrows={-Triangle[angle=90:3pt,blue!70!black,fill=blue!70!black]}] (-1,1) -- (-1,0.5) node [black, text width=0.5cm, below=0mm] {$m_2 \overrightarrow{g}$};
	\draw[fill=red!40!blue] (-2,-0.8) circle (1.5mm);
	\draw[red!40!blue, arrows={-Triangle[angle=90:3pt,red!40!blue,fill=red!40!blue]}] (-2,-0.8) -- (-2,-1.2) node [black, text width=0.5cm, below=-2mm] {$m_3 \overrightarrow{g}$};
	\draw[fill=green!50!red] (1,2) circle (0.8mm);
	\draw[green!50!red, arrows={-Triangle[angle=90:3pt,green!50!red,fill=green!50!red]}] (1,2) -- (1,1.6) node [black, text width=0.5cm, below=-2mm] {$m_1 \overrightarrow{g}$};
\end{tikzpicture}
\caption{Din\'amica interna del molino.}
\label{aletas}
\end{figure}

\subsection{An\'alsis cinem\'atico} \label{cinema}

\noindent
\justify

La trayectoria de las bolas es uno de los par\'ametros que determina la eficiencia del proceso de trituraci\'on, debido a la magnitud de la fuerza y posici\'on del impacto. Este se trata de un an\'alisis simplificado, se desprecia el di\'ametro y el coeficiente de fricci\'on de las bolas, entre otros par\'ametros, como se aprecia en la Figura \ref{simplify}.

\begin{figure}[h!]
\centering
\begin{tikzpicture}	
	\draw (0,0) circle (3cm);
	\draw[arrows={-Triangle[angle=90:2.5pt,black,fill=black]}] (2.1, 2.1) -- (1.5,2.8) node[text width=1mm, above] {$\overrightarrow{V_0}$};	
	\draw[arrows={-Triangle[angle=90:2.5pt,black,fill=black]}] (2.1,2.1) -- (1,2.1) node[text width=1mm, left=3mm] {$V_{0x}$};
	\draw[arrows={-Triangle[angle=90:2.5pt,black,fill=black]}] (2.1,2.1) -- (2.1,2.5) node[text width=1mm, above] {$V_{0y}$};
	\node at (1.6, 2.3) {$\phi$};
	
	%Nomenclatura
	\draw[arrows={-Triangle[angle=90:2.5pt,black,fill=black]}] (0,0) -- (0,1) node[above] {$y$};
	\draw[arrows={-Triangle[angle=90:2.5pt,black,fill=black]}] (0,0) -- (-1,0) node[above] {$x$}; 
	\draw[arrows={-Triangle[angle=90:2.5pt,black,fill=black]}] (3.4,1) -- (3.4,0) node[below] {$\overrightarrow{g}$}; 
	
	
	
	\draw[fill=blue] (2.05,2.05) circle (1mm);
	%\draw[dotted, green!50!red] (2.1, 2.1) -- (-2.1, -2.1);
	\draw[dashed] plot [smooth] coordinates { ( 2.1,2.1) (1.5,1.9) (0, 1) (-1.5,-1.4) (-2,-2) };
	\draw[dotted, blue] (-2.05,-2.05) circle (1mm);
	\draw[red!40!yellow, arrows={-Triangle[angle=90:5pt,red!40!yellow,fill=red!40!yellow]}] (0,3.4) arc(90:120:4.5) node [midway, above] {$\overrightarrow{\omega}$};
	
	\draw[arrows={-Triangle[angle=90:4pt,black,fill=black]}] (0,0) -- (2,2) node[text width=1mm, below] {$\vec{r_0}$};
	
	
\end{tikzpicture}
\caption{Cinem\'atica de una bola (an\'alisis simplificado).}
\label{simplify}
\end{figure}

\noindent
\justify

De la Figura \ref{simplify}, $\vec{V_0}$ es la velocidad inicial, $\phi$ es el \'angulo en el que se distribuye el material a triturar durante el funcionamiento del molino, $\omega$ es la velocidad angular, $\vec{r_0}$ es el vector posici\'on inicial y $\vec{g}$ es el vector aceleraci\'on de la gravedad.

\noindent
\justify

Sabiendo que el valor de la aceleraci\'on en la componente vertical $a_y$ es la gravedad y asumiendo que la aceleraci\'on horizontal tiene un valor constante, e igual a cero $\left( a_x = 0 \right)$, se tiene lo siguiente:

\begin{equation}
\vec{a} = -g \hat{j}
\end{equation}

\noindent
\justify

Sabiendo que la aceleraci\'on es la variaci\'on de la velocidad con respecto al tiempo, se tiene que:

\begin{equation}
\vec{V} = V_{0x} \hat{i} + \left(V_{0y} - gt \right) \hat{j}
\end{equation}

\noindent
\justify

De igual manera, siendo la velocidad la variaci\'on de la posici\'on con respecto al tiempo:

\begin{equation}
\vec{r} = \left(-x_0 + V_{0x} t \right) \hat{i} + \left(y_0 + V_{0y}t - \frac{g t^2}{2} \right) \hat{j}
\end{equation}

\noindent
\justify

La velocidad angular $\omega$ tiene un valor de $3.93 \left(rev /s \right)$ y el radio del molino $r$ es de $0.4065 [m]$, de modo que el valor de la velocidad inicial es de $1.6 [m/s]$. Para una trituraci\'on \'optima, se requiere que el \'angulo de distribuci\'on sea de $45 \degree$, de modo que:

\begin{equation*}
V_{0x} = V_0 \cos \phi \rightarrow V_{0x} = 1.13 [m/s]
\end{equation*}

\begin{equation*}
V_{0y} = V_0 \sin \phi \rightarrow V_{0y} = 1.13 [m/s]
\end{equation*}

\begin{equation}
\vec{V_0} = 1.13 \hat{i} + 1.13 \hat{j} [m/s]
\label{velocidad}
\end{equation}

\noindent
\justify

De igual manera, el vector posici\'on inicial queda de la siguiente forma:

\begin{equation*}
x_{0} = r_0 \cos \phi \rightarrow x_{0} = 0.29 [m]
\end{equation*}

\begin{equation*}
y_{0} = r_0 \sin \phi \rightarrow y_{0} = 0.29 [m]
\end{equation*}

\begin{equation}
\vec{r_0} = 0.29 \hat{i} + 0.29 \hat{j} [m]
\label{posicion}
\end{equation}

\noindent
\justify

Para definir la trayectoria del material dentro del molino, es necesario conocer el tiempo que tarda en realizar su recorrido. Para conocer el tiempo, se debe conocer la posici\'on en donde cae.

\begin{figure}[h!]
\centering
\begin{tikzpicture}
	\draw[arrows={-Triangle[angle=90:4pt,black,fill=black]}] (0,-3) -- (0,3) node[left] {$y$};
	\draw[arrows={-Triangle[angle=90:4pt,black,fill=black]}]  (-3,0) -- (3,0) node[below] {$x$};
	\draw[fill=blue] (2.1,-2.1) circle (0.5mm);
	\draw[fill=blue] (-2.1,2.1) circle (0.5mm);
	
	\draw[dashed, blue!60] (-2.05,2.05) -- (2.05,-2.05) arc(-45:-78:3) -- (0,-2.4) -- (-0.55,-2.85) arc(-102:-168:3) -- (-2.4,0) -- (-2.85,0.55) arc(-192:-225:3) -- cycle;
	\node[below=3mm] at (2.1,-2.1) {$\left(x_f, y_f \right)$};
	\node[above=1mm] at (-2.1,2.1) {$\left(x_0, y_0 \right)$};
	
	\draw[<-] (-1,0) arc(180:135:1);	
	\node[left] at (-1,0.5) {$\phi$};
\end{tikzpicture}
\caption{Tendencia de distribuci\'on del material.}
\label{dist}
\end{figure}

\noindent
\justify

De la Figura \ref{dist}, la distribuci\'on del material est\'a dada por la siguiente relaci\'on matem\'atica:

\begin{equation}
y \left(t_f \right) = -\tan \phi \, x_f \, t_f
\label{tiempo}
\end{equation}

\noindent
\justify

Relacionando las Ecuaciones \ref{posicion} y \ref{tiempo}, se obtiene:

\begin{equation}
t_f ^2 - \frac{2 \left(V_{0y} + V_{0x} \tan \phi \right)}{g} t_f - \frac{2}{g} \left( y_0 - x_0 \tan \phi \right) = 0
\label{tf}
\end{equation}

\noindent
\justify

Relacionando las Ecuaciones \ref{velocidad}, \ref{posicion} y \ref{tf} $t_f$ toma valores de $-0.14 [s]$ y $0.43 [s]$; de modo que una bola llega al punto de impacto en \textbf{\boldsymbol{$0.43 [s]$}}. La relaci\'on entre la posici\'on vertical $y$ con respecto a la posici\'on horizontal $x$ se puede apreciar en la Ecuaci\'on \ref{yx}.

\begin{equation}
y (x) = r \sin \phi + \left(r + \frac{x}{\cos \phi} \right) \sin \phi - \frac{g \left(r + \frac{x}{\cos \phi} \right) ^2}{2 \omega ^2 r^2}
\label{yx}
\end{equation}

\noindent
\justify

La posici\'on de una bola durante la etapa de molienda se puede observar en la Figura \ref{yxgraph}.

\begin{figure}[h!]
\centering
\begin{tikzpicture}
	\begin{axis}[grid = both, minor tick num=2,
			title = \textbf{Trayectoria de una bola},
			xlabel = {$x [m]$},
			ylabel = {$y [m]$}]
	\addplot file {xy.txt};
	\end{axis}
\end{tikzpicture}
\caption{Trayectoria de una bola sobre el molino.}
\label{yxgraph}
\end{figure}

\newpage

\subsection{Impacto}

\noindent
\justify

La frecuencia de impacto de las bolas est\'a definido por la Ecuaci\'on \ref{impacto}.

\begin{equation}
f_i = N_{bi} RPS \, \, \, f_i \in N
\label{impacto}
\end{equation}

\noindent
\justify

D\'onde $N_{bi}$ es el n\'umero de bolas, del tipo de bola $i$ que hay en el molino y $RPS$ es el n\'umero de revoluciones por segundo del molino. La frecuencia de impacto debe ser un n\'umero natural.

\noindent
\justify

Las distribuciones de la energ\'ia de impacto por bola se calcula de acuerdo a lo planteado en la Ecuaci\'on \ref{dist}.

\begin{equation}
Dist_i = \left( \frac{I_i}{\sum _{j=1} ^{N_{tb}} I_j} \right) *100
\label{dist}
\end{equation}

\noindent
\justify

D\'onde $Dist_i$ corresponde a la distribuci\'on de la energ\'ia de impacto del tipo de bola $i$ sobre el molino e $I_i$ a la energ\'ia de impacto de todas las bolas de tipo $i$.




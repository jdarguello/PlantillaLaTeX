\begin{center}
	\section{Separaci\'on de sustancias}
\end{center}

\noindent
\justify

El objetivo de esta etapa es la de separar la mezcla \textit{solvente - extracto}; permitiendo la reutilizaci\'on del solvente para procesos de extracci\'on posteriores. Debido a la degradaci\'on qu\'imica del extracto a temperaturas superiores a los $60 \left[ \degree C \right]$, la separaci\'on se debe dar en condiciones de vac\'io. El sistema a dise\~nar y construir consiste de los siguientes equipos: unidad evaporadora, bomba centr\'ifuga, resistencia el\'ectrica, bomba de vac\'io, v\'alvulas de compuerta y de alivio, y recipientes de agua y solvente; como se aprecia en la Figura \ref{sepa}.

\begin{figure}[h!]
\centering
\begin{minipage}[b]{0.48\textwidth}
	\includegraphics[width=\textwidth]{Images/Separation/Separador.PNG}
\end{minipage}
\hfill
\begin{minipage}[b]{0.48\textwidth}
	\includegraphics[width=\textwidth]{Images/Separation/Separador1.PNG}
\end{minipage}
\caption{Sistema de separaci\'on de sustancias.}
\label{sepa}

\end{figure}

\noindent
\justify

Asumiendo condiciones ambientales de $T = 27 \left[ \degree C \right]$ y $P = 101.325 [KPa]$ (estado 1), el ciclo termodin\'amico del proceso consiste en lo siguiente: 

\begin{enumerate}
	\item Despresurizar la mezcla hasta los $20 [KPa] \rightarrow$ estado 2.
	\item Ebullir el solvente de la mezcla hasta producir vapor saturado $\rightarrow$ estado 3.
	\item Despresurizar el solvente, condens\'andolo en el recipiente contenedor (estado 4).
\end{enumerate}

\newpage

\noindent
\justify

Lo anterior se puede apreciar en la Figura \ref{termo}.

\begin{figure}[h!]
	\centering
	\includegraphics[width=\textwidth]{Images/Separation/diagrama.PNG}
	\caption{Diagrama termmodin\'amico del proceso de recuperaci\'on del solvente.}
	\label{termo}
\end{figure}

\noindent
\justify

El \textit{sistema de separaci\'on} ocupa un espacio de: $5.18 [m]$ de alto, $3 [m]$ de largo y $1 [m]$ de ancho. 
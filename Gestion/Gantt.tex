\subsection{Diagrama de Gantt}

\noindent
\justify

Todo lo explicado anteriormente se puede apreciar gr\'aficamente en la Figura \ref{gantt2}.

\begin{figure}[htp!]
\begin{ganttchart}[
	hgrid, 
	vgrid,
	today=6,
	x unit=4.2mm,
	group/.append style={draw=black, 			fill=green!50},
]{1}{24}
	\gantttitle{\textbf{Cronograma 2020}}{24} \\
	\gantttitlelist{1,...,12}{2}\\
	\ganttgroup[progress=100]{1. PI}{1}{3}\\
	\ganttbar[progress=100]{1.1 BI}{1}{1}\\
	\ganttbar[progress=100]{1.2 RB}{1}{2}\\
	\ganttbar[progress=100]{1.3 SA}{2}{3}\\
	\ganttgroup[progress=100]{2. DC}{4}{6}\\
	\ganttgroup[progress=40]{3. DF}{7}{9} \\
	\ganttgroup[group/.append style={draw=black, fill=yellow!50}]{4. Modelo CFD}{10}{14} \\
	\ganttgroup[group/.append style={draw=black, fill=yellow!50}]{5. PE}{15}{18}\\
	\ganttgroup[group/.append style={draw=black, fill=yellow!50}]{6. PC}{18}{20}\\
	\ganttgroup[group/.append style={draw=black, fill=yellow!50}]{7. RT}{15}{22}
	
\end{ganttchart}
\caption{Cronograma 2020}
\label{gantt2}
\end{figure}
